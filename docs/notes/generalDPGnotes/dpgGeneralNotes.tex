\documentclass{article}

\usepackage{fullpage}
\usepackage{array}
\usepackage{amsmath,amssymb,amsfonts,mathrsfs,amsthm}
\usepackage[utf8]{inputenc}
\usepackage{listings}
\usepackage{mathtools}
\usepackage{pdfpages}
\usepackage[textsize=footnotesize,color=green]{todonotes}
\usepackage{bm}
\usepackage{tikz}
\usepackage[normalem]{ulem}

\usepackage{graphicx}
\usepackage{subfigure}

\usepackage{color}
\usepackage{pdflscape}
\usepackage{pifont}

\usepackage{bibentry}
\nobibliography*

\renewcommand{\topfraction}{0.85}
\renewcommand{\textfraction}{0.1}
\renewcommand{\floatpagefraction}{0.75}

\newcommand{\vect}[1]{\ensuremath\boldsymbol{#1}}
\newcommand{\tensor}[1]{\underline{\vect{#1}}}
\newcommand{\del}{\triangle}
\newcommand{\grad}{\nabla}
\newcommand{\curl}{\grad \times}
\renewcommand{\div}{\grad \cdot}
\newcommand{\ip}[1]{\left\langle #1 \right\rangle}
\newcommand{\eip}[1]{a\left( #1 \right)}
\newcommand{\td}[2]{\frac{d#1}{d#2}}
\newcommand{\pd}[2]{\frac{\partial#1}{\partial#2}}
\newcommand{\pdd}[2]{\frac{\partial^2#1}{\partial#2^2}}

\newcommand{\circone}{\ding{192}}
\newcommand{\circtwo}{\ding{193}}
\newcommand{\circthree}{\ding{194}}
\newcommand{\circfour}{\ding{195}}
\newcommand{\circfive}{\ding{196}}

\newcommand{\Reyn}{\rm Re}

\newcommand{\bs}[1]{\boldsymbol{#1}}
\DeclareMathOperator{\diag}{diag}

\newcommand{\equaldef}{\stackrel{\mathrm{def}}{=}}

\newcommand{\tablab}[1]{\label{tab:#1}}
\newcommand{\tabref}[1]{Table~\ref{tab:#1}}

\newcommand{\theolab}[1]{\label{theo:#1}}
\newcommand{\theoref}[1]{\ref{theo:#1}}
\newcommand{\eqnlab}[1]{\label{eq:#1}}
\newcommand{\eqnref}[1]{\eqref{eq:#1}}
\newcommand{\seclab}[1]{\label{sec:#1}}
\newcommand{\secref}[1]{\ref{sec:#1}}
\newcommand{\lemlab}[1]{\label{lem:#1}}
\newcommand{\lemref}[1]{\ref{lem:#1}}

\newcommand{\mb}[1]{\mathbf{#1}}
\newcommand{\mbb}[1]{\mathbb{#1}}
\newcommand{\mc}[1]{\mathcal{#1}}
\newcommand{\nor}[1]{\left\| #1 \right\|}
\newcommand{\snor}[1]{\left| #1 \right|}
\newcommand{\LRp}[1]{\left( #1 \right)}
\newcommand{\LRs}[1]{\left[ #1 \right]}
\newcommand{\LRa}[1]{\left\langle #1 \right\rangle}
\newcommand{\LRb}[1]{\left| #1 \right|}
\newcommand{\LRc}[1]{\left\{ #1 \right\}}

\newcommand{\Grad} {\ensuremath{\nabla}}
\newcommand{\Div} {\ensuremath{\nabla\cdot}}
\newcommand{\Nel} {\ensuremath{{N^\text{el}}}}
\newcommand{\jump}[1] {\ensuremath{\LRs{\![#1]\!}}}
\newcommand{\uh}{\widehat{u}}
\newcommand{\fnh}{\widehat{f}_n}
\renewcommand{\L}{L^2\LRp{\Omega}}
\newcommand{\pO}{\partial\Omega}
\newcommand{\Gh}{\Gamma_h}
\newcommand{\Gm}{\Gamma_{-}}
\newcommand{\Gp}{\Gamma_{+}}
\newcommand{\Go}{\Gamma_0}
\newcommand{\Oh}{\Omega_h}

\newcommand{\eval}[2][\right]{\relax
  \ifx#1\right\relax \left.\fi#2#1\rvert}

\def\etal{{\it et al.~}}


\def\arr#1#2#3#4{\left[
\begin{array}{cc}
#1 & #2\\
#3 & #4\\
\end{array}
\right]}
\def\vecttwo#1#2{\left[
\begin{array}{c}
#1\\
#2\\
\end{array}
\right]}
\def\vectthree#1#2#3{\left[
\begin{array}{c}
#1\\
#2\\
#3\\
\end{array}
\right]}
\def\vectfour#1#2#3#4{\left[
\begin{array}{c}
#1\\
#2\\
#3\\
#4\\
\end{array}
\right]}

\newcommand{\G} {\Gamma}
\newcommand{\Gin} {\Gamma_{in}}
\newcommand{\Gout} {\Gamma_{out}}


\title{General notes on DPG}
\begin{document}
\maketitle

\section{Least squares}

We begin with a general variational formulation 
\[
b(u,v) = l(v)
\]

DPG begins with the idea that you would like to do least squares on the operator equation 
\[
Bu = \ell, \quad Bu, \ell \in V'
\]
where $\LRa{Bu,v}_{V'\times V} = b(u,v)$ and $\LRa{\ell, v}_{V'\times V} = l(v)$.  Since $Bu - \ell \in V'$, we minimize the norm of this residual in $V'$ over the finite dimensional space $U_h$, i.e. 
\[
\min_{u_h \in U_h} \nor{Bu_h - \ell}_{V'}^2.
\]
This leads to the normal equations 
\[
\LRp{Bu-\ell, B\delta u}_{V'}, \quad \forall \delta u \in U_h.
\]
The Riesz map gives us the equivalent definition 
\[
\LRp{R_V^{-1}\LRp{Bu-\ell}, R_V^{-1}\LRp{B\delta u}}_{V} = 0, \quad \forall \delta u \in U_h.
\]
Assuming we've specified the Riesz map through a test space inner product
\[
\LRa{R_V v , \delta v}_{V'\times V}= (v,\delta v)_V,
\]
this leads to what I call a Dual Petrov-Galerkin method.  

\section{Algebraic perspective} 

In the above example, $V$ is infinite dimensional.  If we approximate $V$ by $V_h$ such that $\dim (V_h) > \dim (U_h)$, we get matrix representations of our operators
\begin{align*}
B_{ij} &= b(u_j, v_i), \quad u_j \in U_h, v_i \in V_h\\
R_V &= (v_i,v_j)_V \quad  v_i,v_j\in V_h\\
\ell_i &= l(v_i), \quad v_i \in V_h.
\end{align*}
The resulting normal equations 
\[
\LRp{R_V^{-1}\LRp{Bu-\ell}, R_V^{-1}\LRp{B\delta u}}_{V}, \quad \forall \delta u \in U_h.
\]
can now be written as 
\[
\LRp{R_V^{-1}\LRp{Bu-\ell}}^T R_V \LRp{R_V^{-1}B} = 0, 
\]
or, after simplifying to $\LRp{Bu-\ell}^T R_V^{-1} B = 0$, we get the algebraic normal equations
\[
B^TR_V^{-1}Bu = B^TR_V^{-1} \ell.
\]
This is just the solution to the algebraic least squares problem
\[
\min_u \nor{Bu-\ell}_{R_V^{-1}}^2.
\]
Such problems can also be written using the augmented system for the least squares problem
\[
\arr{R_V}{B}{B^T}{0}\vecttwo{e}{u} = \vecttwo{l}{0}.
\]
This can be interpreted as the mixed form of the Dual Petrov-Galerkin method, which is used by Cohen, Welper, and Dahmen in their 2012 paper ``Adaptivity and variational stabilization for convection-diffusion equations''.  
\begin{align*}
(e,v)_V + b(u,v) &= l(v) \\
b(\delta u,e) &= 0
\end{align*}
Eliminating $e$ from the above system leads to the above algebraic normal equations.  

\emph{Note: for general $R_V$, the algebraic normal equations are completely dense.  Cohen, Welper, and Dahmen thus solve the augmented system to get solutions in this setting; however, this is a saddle point problem, and over 2x as large as the trial space, which makes preconditioning and solving more difficult. }

\section{Deriving the Discontinuous Petrov-Galerkin method}
We want to avoid solving either a fully dense system or a saddle point problem, so we introduce Lagrange multipliers $\uh$ to enforce continuity weakly on $e$, which we will now approximate using discontinuous functions.  These $\uh$ are defined on element edges only, similarly to hybrid variables or mortars in finite elements.  This leads to the new system 
\begin{align*}
\LRa{\uh, \jump{v}}_{\Gh} + (e,v)_V + b(u,v) &= l(v) \\
b(\delta u,e) &= 0\\
\LRa{\hat{\mu}, \jump{e}}_{\Gh} &= 0
\end{align*}
where $\Gh$ is the mesh skeleton (union of all element edges).  The resulting algebraic system here is 
\[
\left[
\begin{array}{c c c}
R_V & B& \hat{B}\\
B^T & 0& 0\\
\hat{B}^T & 0& 0
\end{array}
\right]
\left[\begin{array}{c}
e \\ u\\ \uh
\end{array}
\right]
= 
\left[\begin{array}{c}
f\\0\\0
\end{array}
\right].
\]
Because $e$ is now discontinuous, $R_V$ is made block-diagonal; eliminating $e$ returns the (fairly) sparse symmetric positive-definite DPG system.


\end{document}

